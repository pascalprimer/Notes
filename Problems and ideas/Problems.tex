\documentclass[12pt]{article}

\usepackage{../pascal}

\begin{document}

\section{Learning sparse monomials with additive error}
\begin{problem}
  Given distribution $\mathcal{D}$ on $R^p\,,$ learning a monomial
  $p(x) = c \sum_{k \in S} x_k^{\beta_k} + \eta\,,$ where $c$ is a
   normalized constant such that $\Ex_{x \sim \mathcal{D}}\left[c^2
   \sum_{k \in S} x_k^{2\beta_k} \right] = 1\,,$ $\eta$ is some noise,
   and $\vert S \vert = k\,,$ with sampling complexity poly($k\,,p$) and
   polynomial running time.
\end{problem}
\begin{thoughts}
  "Attribute-efficient learning of monomials over highly-correlated
  variables" is about the unnoised version.
\end{thoughts}

\section{Improving or giving lowerbounds of learning $k$-sparse,
  $d$-degree polynomials}
\begin{problem}
  As the title. Note that this $k$-sparse is based on the orthonormal
  basis in the given distribution.
\end{problem}
\begin{thoughts}
  "Learning Sparse Polynomial Functions" is about this topic. Can we improve the $\exp(d)$ factor or prove that it is actually essential?

  \textbf{\textit{What is the desired definition of SQ dimension?}} Is
  the one in "https://arxiv.org/pdf/1611.03473.pdf" useful?

  First, read and try to understand "Gradient Descent for One-Hidden-
  Layer Neural Networks: Polynomial Convergence and SQ Lower Bounds" and
  "Statistical Algorithms and a Lower Bound for Detecting Planted
  Cliques". After that, think about if bounded range is necessary or how
  to get rid of it. May truncate the too large samples is a method since
  \textbf{\textit{in all algs the query function have a exp decay}.



  Can we reduce general queries to dot product queries if approximation
  permitted?

  Can we follow the proof in "Weakly Learning DNF and Characterizing Statistical Query Learning Using Fourier Analysis" to some extent?


\end{thoughts}


\end{document}
