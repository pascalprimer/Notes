\documentclass[12pt]{article}

%   Packages

\usepackage[utf8]{inputenc}
\usepackage[T1]{fontenc}
\usepackage[sc]{mathpazo}
\usepackage{microtype}
\usepackage{amsmath}
\usepackage{amssymb}
\usepackage{amsthm}
\usepackage{bm}
\usepackage{bbm}
\usepackage{dsfont}
\usepackage{authblk}
\usepackage{fullpage}
\usepackage{comment}
\usepackage{tikz}
\usepackage{mathtools}
\usepackage[shortlabels]{enumitem}
\usepackage{complexity}
\usepackage{hyperref}
\usepackage[capitalize, nameinlink]{cleveref}
\crefname{ineq}{inequality}{inequalities}
\creflabelformat{ineq}{#2{\upshape(#1)}#3}
\usepackage{nag}

%   Styles

\setlength{\textwidth}{6.3in}
\setlength{\textheight}{8.7in}
\setlength{\topmargin}{0pt}
\setlength{\headsep}{0pt}
\setlength{\headheight}{0pt}
\setlength{\oddsidemargin}{0pt}
\setlength{\evensidemargin}{0pt}

%   Theorems

\theoremstyle{plain}
\newtheorem{theorem}{Theorem}[section]
\newtheorem{lemma}[theorem]{Lemma}
\newtheorem{corollary}[theorem]{Corollary}
\newtheorem{proposition}[theorem]{Proposition}
\newtheorem{fact}[theorem]{Fact}
\newtheorem{conjecture}[theorem]{Conjecture}

\theoremstyle{definition}
\newtheorem{definition}[theorem]{Definition}

\theoremstyle{remark}
\newtheorem{remark}[theorem]{Remark}
\newtheorem{example}[theorem]{Example}

%   Macros

\newcommand*{\Q}{{\mathbb{Q}}}
\newcommand*{\N}{{\mathbb{N}}}
\newcommand*{\Z}{{\mathbb{Z}}}
\let\R\relax
\newcommand*{\R}{{\mathbb{R}}}
\newcommand*{\F}{{\mathbb{F}}}
\let\C\relax
\newcommand*{\C}{{\mathbb{C}}}
\newcommand*{\bigsum}{\sum\limits}
\newcommand*\diff{\mathop{}\!\mathrm{d}}
\newcommand*{\lf}{\mathop{\lfloor\!}}
\newcommand*{\rf}{\mathop{\!\rfloor}}
\newcommand*{\Per}{\operatorname{Per}}
\let\Pr\relax
\newcommand*{\Pr}{\mathbb{P}}
\let\Ex\relax
\newcommand*{\Ex}{\mathbb{E}}


\let\poly\relax
\DeclareMathOperator{\poly}{poly}
\let\Im\relax
\DeclareMathOperator{\Im}{Im}
\let\Re\relax
\DeclareMathOperator{\Re}{Re}
\DeclareMathOperator{\Var}{Var}
\DeclareMathOperator{\Cov}{Cov}
\providecommand{\norm}[1]{\left\lVert#1\right\rVert}

\newcommand{\doi}[1]{\href{http://dx.doi.org/#1}{\texttt{doi:#1}}}
\newcommand{\arxiv}[1]{\href{http://arxiv.org/abs/#1}{\texttt{arXiv:#1}}}

\begin{document}

\section{Orlicz Norms and its Generalization}
\begin{definition}[Orlicz Norms]
  Let $g:[0, \infty) \rightarrow [0, \infty)$ be a non-decreasing function
  with $g(0) = 0\,.$ of a real-valued random variable $X$ is given by
  \begin{equation}\label{def-orlicz-norm}
    \left\lVert X \right\rVert_g := \inf\{\eta > 0: \Ex\left[g(\vert X /\eta
    \vert) \right] \le 1\}\,.
  \end{equation}
\end{definition}

It follows from (\ref{def-orlicz-norm}) that if $g$ is monotone,
\begin{equation} \nonumber
  \Pr\!\left(\vert X \vert \ge \eta g^{-1}(t) \right) \le \dfrac{1}{t} \text{
  for all } t \ge 0\,.
\end{equation}

\begin{definition}[Sub-Weibull Variable]
  A random variable $X$ is said to be sub-Weibull of order $\alpha > 0\,$,
  denoted as sub-Weibull ($\alpha$), if
  \begin{equation} \label{def-sub-weibull}
    \norm{X}_{\psi_\alpha} < \infty, \text{ where } \psi_\alpha(x) ~:=~
    \exp\left(x^\alpha \right) - 1 \text{  for } x \ge 0\,.
  \end{equation}
\end{definition}
Based on this definition, it follows that if $X$ is sub-Weibull ($\alpha$), then
\begin{equation}\nonumber
  \Pr\!\left(\vert X \vert \ge t \right) \le 2 \exp\!\left(-
  \dfrac{t^\alpha}{\norm{X}^\alpha_{\psi_{\alpha}}} \right) \text{ for all } t
  \ge 0\,.
\end{equation}
Typically, $X$ is sub-gaussian when $\alpha = 2$ and is sub-exponential when $
\alpha = 1\,.$

\begin{definition}[Marginal Sub-Weibull Vectors]
  A random vector $X\in\mathbb{R}^q$ is said to be marginally sub-Weibull if for
  every $1\le j\le q$, $X(j)$ is sub-Weibull and the marginal sub-Weibull norm
  is given by
  \[
  \norm{X}_{M,\psi_{\alpha}} ~:=~ \sup_{1\le j\le q}\,\norm{X(j)}_{\psi_{\alpha}}.
  \]
\end{definition}

\end{document}
