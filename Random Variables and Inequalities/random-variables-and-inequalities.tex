\documentclass[12pt]{article}

\usepackage{../pascal}

\begin{document}

\section{Orlicz Norms and its Generalization}
\begin{definition}[Orlicz Norms]
  Let $g:[0, \infty) \rightarrow [0, \infty)$ be a non-decreasing function
  with $g(0) = 0\,.$ of a real-valued random variable $X$ is given by
  \begin{equation}\label{def-orlicz-norm}
    \left\lVert X \right\rVert_g := \inf\{\eta > 0: \Ex\left[g(\vert X /\eta
    \vert) \right] \le 1\}\,.
  \end{equation}
\end{definition}

It follows from (\ref{def-orlicz-norm}) that if $g$ is monotone,
\begin{equation} \nonumber
  \Pr\!\left(\vert X \vert \ge \eta g^{-1}(t) \right) \le \dfrac{1}{t} \text{
  for all } t \ge 0\,.
\end{equation}

\begin{definition}[Sub-Weibull Variable]
  A random variable $X$ is said to be sub-Weibull of order $\alpha > 0\,$,
  denoted as sub-Weibull ($\alpha$), if
  \begin{equation} \label{def-sub-weibull}
    \norm{X}_{\psi_\alpha} < \infty, \text{ where } \psi_\alpha(x) ~:=~
    \exp\left(x^\alpha \right) - 1 \text{  for } x \ge 0\,.
  \end{equation}
\end{definition}
Based on this definition, it follows that if $X$ is sub-Weibull ($\alpha$), then
\begin{equation}\nonumber
  \Pr\!\left(\vert X \vert \ge t \right) \le 2 \exp\!\left(-
  \dfrac{t^\alpha}{\norm{X}^\alpha_{\psi_{\alpha}}} \right) \text{ for all } t
  \ge 0\,.
\end{equation}
Typically, $X$ is sub-gaussian when $\alpha = 2$ and is sub-exponential when $
\alpha = 1\,.$

\begin{definition}[Marginal Sub-Weibull Vectors]
  A random vector $X\in\mathbb{R}^q$ is said to be marginally sub-Weibull if for
  every $1\le j\le q$, $X(j)$ is sub-Weibull and the marginal sub-Weibull norm
  is given by
  \[
  \norm{X}_{M,\psi_{\alpha}} ~:=~ \sup_{1\le j\le q}\,\norm{X(j)}_{\psi_{\alpha}}.
  \]
\end{definition}

\end{document}
