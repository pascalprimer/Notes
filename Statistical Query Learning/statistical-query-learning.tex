\documentclass[12pt]{article}

\usepackage{../pascal}
\DeclareMathOperator{\WSQD}{SQDim}

\begin{document}

\section{Philosophy}
\begin{itemize}
  \item Learning algorithms are to compute parameters while complexity
    analysis focuses on distinguishing different functions from some
    concept class.
\end{itemize}

\section{Weak SQ Dimension}
\begin{definition}
  The (weak) SQ dimension of a class of real valued functions
  $\mathcal{F}$ over some domain $X$ under distribution $\mathcal{D}$
  , denoted $\WSQD_{\mathcal{F}}^\mathcal{D}$, is the biggest $d$ such
  that $\mathcal{F}$ contains distinct $f_1\,, \cdots\,, f_d$ with
  pairwise correlations between $-\frac{1}{d}$ and $\frac{1}{d}\,.$
  Note that correlation between $f$ and $g$ is defined by
  \begin{equation}  \nonumber
    \langle f, g \rangle \triangleq \Ex_{x \sim \mathcal{D}} f(x)g(x)\,.
  \end{equation}
\end{definition}

\begin{example}
  When consider $\mathcal{F}$ as set of vectors on $S^{n - 1}$, inner
  product as standard inner product and $\mathcal{D}$ as uniform
  distribution, $\WSQD$ is now polynomial; while $\WSQD$ is $2^n$ for
  parity functions under uniform distribution.
\end{example}

\end{document}
